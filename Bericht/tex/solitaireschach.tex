\documentclass[
	12pt,
	a4paper,
	BCOR10mm,
	%chapterprefix,
	DIV14,
	listof=totoc,
	bibliography=totoc,
	headsepline
]{scrreprt}

\usepackage[T1]{fontenc}
\usepackage[utf8]{inputenc}
\usepackage{ngerman}

\usepackage{lmodern}

\usepackage[footnote]{acronym}
\usepackage[page,toc]{appendix}
\usepackage{fancyhdr}
\usepackage{float}
\usepackage{graphicx}
\usepackage[pdfborder={0 0 0}]{hyperref}
\usepackage[htt]{hyphenat}
\usepackage{listings}
\usepackage{lscape}
\usepackage{microtype}
\usepackage{nicefrac}
\usepackage{subfig}
\usepackage{textcomp}
\usepackage[subfigure,titles]{tocloft}
\usepackage{units}

\lstset{
	basicstyle=\ttfamily,
	frame=single,
	numbers=left,
	language=C,
	breaklines=true,
	breakatwhitespace=true,
	postbreak=\hbox{$\hookrightarrow$ },
	showstringspaces=false,
	tabsize=4
}

\renewcommand*{\lstlistlistingname}{Listingverzeichnis}

\renewcommand*{\appendixname}{Anhang}
\renewcommand*{\appendixtocname}{Anhänge}
\renewcommand*{\appendixpagename}{Anhänge}

\begin{document}

\begin{titlepage}
	\begin{center}
		{\titlefont\huge Solitaire-Schach\par}

		\bigskip
		\bigskip

		{\titlefont\Large --- Praktikumsbericht ---\par}

		\bigskip
		\bigskip

		{\large Arbeitsbereich Wissenschaftliches Rechnen\\
		Fachbereich Informatik\\
		Fakultät für Mathematik, Informatik und Naturwissenschaften\\
		Universität Hamburg\par}
	\end{center}

	\vfill

	{\large \begin{tabular}{ll}
		Vorgelegt von: & Kira Duwe\\
		E-Mail-Adresse: & \href{mailto:adresse@email.de}{0duwe@informatik.uni-hamburg.de} \\
		Matrikelnummer: & 6225091 \\
		Studiengang: & Informatik \\
		\\
			Vorgelegt von: & Enno Zickler\\
		E-Mail-Adresse: & \href{mailto:adresse@email.de}{0zickler@informatik.uni-hamburg.de} \\
		Matrikelnummer: & 6250134 \\
		Studiengang: & Informatik \\
		\\
		%Erstgutachter: & Name des Erstgutachters \\
		%Zweitgutachter: & Name des Zweitgutachters\\ \\
		Betreuer: & Julian Kunkel \\
		\\
		Hamburg, den 30.09.2013
	\end{tabular}\par}
\end{titlepage}

\chapter*{Abstract}

\thispagestyle{empty}

TODO: Hier kommt eine kurze Beschreibung der nachfolgenden Arbeit hin.

%Inhaltsverzeichnis
\tableofcontents

%Kapitel 1
\chapter{Einleitung}
\label{Einleitung}

\section{Aufgabenstellung}

\section{Spielregeln}

\section{Begrifflichkeiten}
Hier eine Definition der im folgenden gebrauchten Begriffe:
\begin{itemize}
	\item Spielbrett: die aktuelle Belegung der einzelnen Felder
	\item Spielfeld: genau ein Einzelsegment des Spielbrettes
	\item Spielbrettbreite: x-Koordinate des Spielbrettes
	\item Spielbretthöhe: y-Koordinate des Spielbrettes
	\item Felderanzahl= Spielbrettbreite x Spielbretthöhe
\end{itemize}


%Kapitel 2
\chapter{Entwurf \& Implementierung}
\label{Entwurf}

\section{Entwurf}
\section{Probleme}
\subsection{Spielbretterzeugung}
\subsection{Speicherung der Spielbretter}
\subsection{Berechnung der Spielbretter}
\subsection{Lastungleichheit}

\textit{%
In diesem Kapitel ...
}
\bigskip

\section{Implementierung}
\subsection{Darstellung der Spielbretter}
\subsection{Erzeugung der Spielbretter}
\subsection{Speicherung der Spielbretter}
\subsection{Parallelisierung}
\subsection{Kommunikation}


%Kapitel 3
\chapter{Ergebnisse}
\label{Ergebnisse}

\section{Laufzeiten}

\section{Lösungen pro Sekunde}

\section{Hashtablegröße}

\section{Speedup}

\section{Lastenverteilung}


%Kapitel 4
\chapter{Fazit}
\label{Fazit}


%Letzte Seite!
\newpage

\thispagestyle{empty}

\chapter*{}

\section*{Erklärung}

Ich versichere, dass ich die Arbeit selbstständig verfasst und keine anderen, als die angegebenen Hilfsmittel -- insbesondere keine im Quellenverzeichnis nicht benannten Internetquellen -- benutzt habe, die Arbeit vorher nicht in einem anderen Prüfungsverfahren eingereicht habe und die eingereichte schriftliche Fassung der auf dem elektronischen Speichermedium entspricht.

\smallskip

\bigskip
\bigskip
\bigskip

Hamburg, den 30.09.2013  \quad \dotfill

\end{document}
